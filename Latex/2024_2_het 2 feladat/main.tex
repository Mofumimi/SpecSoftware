\documentclass[12pt]{article}
\usepackage{fancyhdr}
\usepackage{enumitem}
\usepackage{hyperref}
\usepackage{lipsum}
\usepackage{xcolor}
\usepackage{geometry}
\usepackage{varioref}
\usepackage{cleveref}

\hypersetup{
    colorlinks=true,
    linkcolor=blue,
    citecolor=red,
    urlcolor=cyan
}

\setlength{\headheight}{15pt}

\title{2. Feladat}
\author{Gecse Tamás}
\date{}

\begin{document}

\maketitle
\tableofcontents
\newpage

% 1. Feladat
\section{1. Feladat (Kereszthivatkozások)}

% b) feladat
\lipsum[1] 
\label{sec:label_zagyva} 
\lipsum[2]
\subsection{Mélyebb szakasz}
\lipsum[3]
\subsubsection{Még mélyebb szakasz}
\label{sec:deep} 
\lipsum[4]

\appendix
\section{Függelék}
\label{sec:appendix}
\lipsum[5]

\section{Leíró lista}
\begin{description}
    \item[Elem 1] \label{desc-item} Leírás az első elemhez.
    \item[Elem 2] Leírás a második elemhez.
    \item[Elem 3] Leírás a harmadik elemhez.
\end{description}

\section{Számozott lista}
\begin{enumerate}
    \item Első elem \label{numbered-item}
    \item Második elem
    \item Harmadik elem
\end{enumerate}

\begin{enumerate}
    \item[] Számozatlan elem \label{unnumbered-item}
\end{enumerate}

% c) feladat
\newpage
\section{Kereszthivatkozások tesztelése}
Az alábbiakban különböző kereszthivatkozásokat találhatunk:
\begin{itemize}
    \item Hivatkozás zagyva szövegre: \ref{sec:label_zagyva}, \pageref{sec:label_zagyva}. oldal.
    \item Hivatkozás mélyebb szakaszra: \ref{sec:deep}, \pageref{sec:deep}. oldal.
    \item Hivatkozás a függelékhez: \ref{sec:appendix}, \pageref{sec:appendix}. oldal.
    \item Hivatkozás leíró lista elemére: \ref{desc-item}, \pageref{desc-item}. oldal.
    \item Hivatkozás számozott lista elemére: \ref{numbered-item}, \pageref{numbered-item}. oldal.
    \item Hivatkozás számozatlan elemre: \ref{unnumbered-item}, \pageref{unnumbered-item}. oldal.
\end{itemize}

% d) feladat

% e) feladat
Weboldal hivatkozás: \href{https://www.example.com}{Példa Weboldal}

% f) feladat

% g) feladat
Az \autoref parancs: \autoref{sec:deep}.
Az \autopageref parancs: \autopageref{sec:deep}.

\newpage
% 2. Feladat 
\section*{2. Feladat (Számozatlan szakasz)}
\label{sec:szamozatlan}
\lipsum[1-8] 

\newpage
% 3. Feladat (\input vs \include)
\section{3. Feladat (\\input vs \\include)}
\section{Első Szakasz}
\label{sec:first}

Ez az első szakasz, amely tartalmaz egy label-t. Itt van egy kis zagyva szöveg is:

\lipsum[1]

\subsection{Alszakasz}
\label{sec:sub}

Ez az alszakasz a `section1.tex` fájlban van, és tartalmaz egy label-t is.

\lipsum[2]
\section{Második Szakasz}
\label{sec:second}

Ez a második szakasz, és itt van egy hivatkozás az első szakasz label-jére: \ref{sec:first}.

\lipsum[3]

\newpage
% 4. Szorgalmi feladat (Hivatkozások automatizálása)
\section{4. Szorgalmi feladat (Hivatkozások automatizálása)}
Az \texttt{varioref} csomag hivatkozása: \vref{sec:deep}.
Az \texttt{cleveref} csomag használata: \cref{sec:deep}.

\end{document}