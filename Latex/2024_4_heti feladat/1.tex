\documentclass{article}
\usepackage{amsthm}
\usepackage{amsmath}
\usepackage{amssymb}

% Tétel és Lemma környezetek definiálása
\newtheorem{theorem}{Tétel}
\newtheorem{lemma}[theorem]{Lemma}

% Definíció section-önként számozva
\theoremstyle{definition}
\newtheorem{definition}{Definíció}[section]

% Makrók
\DeclareMathOperator*{\argmax}{arg\,max}
\newcommand{\E}[1]{\mathbb{E}\left[#1\right]}
\newcommand{\CE}[2]{\mathbb{E}\left[#1 \mid #2\right]}

\begin{document}

% 1. Feladat (Tételkörnyezetek)
\section{Tételkörnyezetek}

\begin{theorem}
Ez egy névtelen tétel.
\end{theorem}

\begin{theorem}[Példa szerző]
Ez egy tétel szerzővel.
\end{theorem}

\begin{lemma}
Ez egy Lemma.
\end{lemma}

\begin{proof}
Ez a tétel bizonyítása.
\end{proof}

\begin{proof}[Lemma bizonyítása]
Ez a bizonyítás hivatkozik a Lemmára.
\end{proof}

% 2. Feladat (Matematikai formulák A-tól Z-ig)
\section{Matematikai formulák}

Az \( \frac{1}{n^2} \) sorösszege a következő:

\[
\sum_{n=1}^{\infty} \frac{1}{n^2} = \frac{\pi^2}{6}.
\]

A faktoriális definíciója:

\[
n! := \prod_{k=1}^{n} k = 1 \cdot 2 \cdot \ldots \cdot n.
\]

Legyen \( 0 \leq k \leq n \). A binomiális együttható:

\[
\binom{n}{k} := \frac{n!}{k!(n-k)!}.
\]

Az előjel- (szignum) függvény a következő:

\[
\text{sgn}(x) := 
\begin{cases} 
1 & \text{ha } x > 0, \\
0 & \text{ha } x = 0, \\
-1 & \text{ha } x < 0.
\end{cases}
\]

% 3. Szorgalmi feladat (Matematikai makrók)
\section{Matematikai makrók}

Az \texttt{arg max} operátor definíciója:

\[
x^* := \argmax_{x \in [0,1]} x \log_2(x)
\]

A felső egészrész függvény:

\[
\left\lceil \frac{5}{3} \right\rceil
\]

Várható érték makró használata:

\[
\E{X_i}
\]

Feltételes várható érték makró használata:

\[
\CE{X_i}{X_j}
\]

Toronyszabály formulája:

\[
\mathbb{E}\left[\sum_{i=1}^N X_i\right] = \mathbb{E}\left[\mathbb{E}\left[\sum_{i=1}^N X_i \mid N\right]\right]
\]

% 4. Feladat (Binomiális tétel)
\section{Binomiális tétel}

A binomiális tétel induktív bizonyítása align és split használatával:

\begin{align}
(a+b)^{n+1} &= (a+b) \cdot \left( \sum_{k=0}^{n} \binom{n}{k} a^{n-k} b^k \right) \label{eq:binom1}\\
&= \sum_{k=0}^{n} \binom{n}{k} a^{(n+1)-k} b^k + \sum_{k=1}^{n+1} \binom{n}{k-1} a^{(n+1)-k} b^k \label{eq:binom2}\\
&= \binom{n+1}{0} a^{n+1} b^0 + \sum_{k=1}^{n} \binom{n+1}{k} a^{(n+1)-k} b^k + \binom{n+1}{n+1} a^0 b^{n+1} \label{eq:binom3}\\
&= \sum_{k=0}^{n+1} \binom{n+1}{k} a^{(n+1)-k} b^k \label{eq:binom4}
\end{align}

\end{document}
