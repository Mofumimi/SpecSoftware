\documentclass[12pt, aspectratio=169]{beamer}
\PassOptionsToPackage{table}{xcolor}

\title{0. Dia}
\author{Gecse Tamás}
\date{\today}

\begin{document}

% 1. Feladat h) Címdia a dokumentum elején
\begin{frame}
    \titlepage
\end{frame}

% 3. Feladat a) Section 1
\section{Első Feladatsor}

% Tartalomjegyzék a section elején
\begin{frame}
    \frametitle{Tartalomjegyzék}
    \tableofcontents[currentsection, hideothersubsections]
\end{frame}

% 1. Feladat d) Frame cím és alcím
\begin{frame}
    \frametitle{Frame címe}
    \framesubtitle{Frame alcíme}
    Néhány szó a tartalomról.
\end{frame}

% 1. Feladat e) Több dummy frame
\begin{frame}
    \frametitle{Második Frame}
    Itt egy másik tartalom található.
\end{frame}

\begin{frame}
    \frametitle{Harmadik Frame}
    Egy újabb tartalom.
\end{frame}

% 1. Feladat f) Verbatim szöveg elhelyezése
\begin{frame}[fragile]
    \frametitle{Verbatim Példa}
    \begin{verbatim}
        Ez egy verbatim szöveg.
    \end{verbatim}
\end{frame}

% 1. Feladat g) Több bekezdésnyi zagyva szöveg és automatikus tördelés
\begin{frame}[allowframebreaks]
    \frametitle{Több bekezdés}
    Lorem ipsum dolor sit amet, consectetur adipiscing elit. Sed do eiusmod tempor incididunt ut labore et dolore magna aliqua. 
    
    Ut enim ad minim veniam, quis nostrud exercitation ullamco laboris nisi ut aliquip ex ea commodo consequat.
    
    Duis aute irure dolor in reprehenderit in voluptate velit esse cillum dolore eu fugiat nulla pariatur.
\end{frame}

% 2. Feladat a) Kétoszlopos dia
\begin{frame}
    \frametitle{Kétoszlopos Dia}
    \begin{columns}
        \column{0.5\textwidth}
        \begin{itemize}
            \item Első pont
            \item Második pont
        \end{itemize}
        \begin{enumerate}
            \item Első elem
            \item Második elem
        \end{enumerate}
        
        \column{0.5\textwidth}
        \includegraphics[width=\textwidth]{kep.png}
        \caption{Kép felirattal}
    \end{columns}
\end{frame}

% 2. Feladat b) block, exampleblock, alertblock
\begin{frame}
    \frametitle{Block Tesztelés}
    
    \begin{block}{Block példa}
        Ez egy normál block.
    \end{block}
    
    \begin{exampleblock}{Példa block}
        Ez egy exampleblock.
    \end{exampleblock}
    
    \begin{alertblock}{Figyelmeztető block}
        Ez egy alertblock.
    \end{alertblock}
    
    \begin{block}{}
        Ez egy cím nélküli block.
    \end{block}
\end{frame}

% 2. Feladat c) theorem és proof
\begin{frame}
    \frametitle{Tétel és Bizonyítás}
    \begin{theorem}
        Ez egy tétel.
    \end{theorem}
    
    \begin{proof}
        Ez a bizonyítás szövege.
    \end{proof}
\end{frame}

% 2. Feladat d) semiverbatim használata
\begin{frame}[fragile]
    \frametitle{Semiverbatim Példa}
    \begin{semiverbatim}
        \textcolor{red}{\LaTeX} kód példa itt.
    \end{semiverbatim}
\end{frame}

% 3. Feladat a) Section 2
\section{Második Feladatsor}

\begin{frame}
    \frametitle{Tartalomjegyzék}
    \tableofcontents[currentsection, hideothersubsections]
\end{frame}

% 4. Feladat a) Beamer theme és color theme kombinációk
\begin{frame}
    \frametitle{Beamer Theme Teszt}
    Ez a frame a különböző beamer témák tesztelésére készült.
\end{frame}

% 5. Feladat a) \pause parancs tesztelése
\begin{frame}
    \frametitle{Pause Teszt}
    Első pont.
    \pause
    Második pont.
    \pause
    Harmadik pont.
\end{frame}

% 5. Feladat b) \item<> tesztelése
\begin{frame}
    \frametitle{Elemek megjelenítése sorrendben}
    \begin{itemize}
        \item<3> Első elem
        \item<2> Második elem
        \item<1> Harmadik elem
    \end{itemize}
\end{frame}

% 5. Feladat c) Tétel és bizonyítás késleltetett megjelenítése
\begin{frame}
    \frametitle{Tétel Késleltetve}
    \begin{theorem}
        Ez egy újabb tétel.
    \end{theorem}
    \pause
    \begin{proof}
        A bizonyítás csak később jelenik meg.
    \end{proof}
\end{frame}

% 5. Feladat e) Képek váltása ugyanazon a helyen
\begin{frame}
    \frametitle{Képek váltása}
    \only<1>{\includegraphics[width=\textwidth]{getto.png}}
    \only<2>{\includegraphics[width=\textwidth]{1.jpg}}
\end{frame}

% 6. Feladat a) Áttűnés tesztelése
\begin{frame}
    \frametitle{Áttűnés Teszt}
    Különböző áttűnések frame-ek között.
\end{frame}

\end{document}
